\documentclass{article}
% Margin definition.
\usepackage[a4paper,total={6.8in, 8.5in}]{geometry}
\usepackage{parskip}
\usepackage[bottom]{footmisc}
% Images.
\usepackage{graphicx}
\usepackage[hidelinks, bookmarks=true]{hyperref}
\usepackage{float}
% Encoding.
\usepackage[english]{babel}
\usepackage[utf8]{inputenc}
% To have another layer of sub sections - \paragraph
\usepackage{titlesec}

\setcounter{secnumdepth}{4}

\titleformat{\paragraph}
{\normalfont\normalsize\bfseries}{\theparagraph}{1em}{}
\titlespacing*{\paragraph}
{0pt}{3.25ex plus 1ex minus .2ex}{1.5ex plus .2ex}
% Allow multiline comments
\usepackage{verbatim} 
% Helvetic font.
\usepackage[scaled]{helvet}
\renewcommand\familydefault{\sfdefault} 
% Header for UA logo.
\usepackage{fancyhdr}
% Dots in index.
\usepackage[titles]{tocloft}
\renewcommand{\cftsubsecleader}{\Large\cftdotfill{0}}
\renewcommand{\cftsecleader}{\Large\cftdotfill{0}}
\renewcommand{\cftsecfont}{\large\bfseries\scshape}
\renewcommand{\cftsubsecfont}{\scshape}
\renewcommand*{\HyperDestNameFilter}[1]{\jobname-#1}
% Dot after number in (sub)sections and in toc.
\renewcommand{\cftsecaftersnum}{.}
\renewcommand{\cftsubsecaftersnum}{.}
\usepackage{titlesec}
\titlelabel{\hspace{-0.5cm}\quad}
\usepackage[letterspace=45]{microtype}
\newcommand*{\fullref}[1]{\hyperref[{#1}]{\autoref*{#1}

\nameref*{#1}}}
% Header with UA logo definition. 
\pagestyle{fancy}
\fancyhf{}
\chead{
    \includegraphics[width=5in]{./images/header_ua.png}
}
\setlength\headheight{45pt}
\setlength\parindent{24pt}
\usepackage{indentfirst}
% Footer with page number.
\rfoot{Page \thepage}
\renewcommand{\footrulewidth}{0.1pt}

% Rename table of contents title to "Index"
\renewcommand{\contentsname}{\normalsize Index \vspace{0.6cm}}
% Add text with hyperlink
\usepackage{hyperref}
%\hypersetup{
%    colorlinks=true,
%    linkcolor=blue,
%    filecolor=magenta
%}
% Water mark
\newsavebox\mybox
\usepackage[printwatermark]{xwatermark}
\usepackage{xcolor}
\usepackage{tikz}
% paragraph
\newcommand\tab[1][1cm]{\hspace*{#1}}
\setlength\parindent{24pt}
%images
 \usepackage{graphicx}
\usepackage{caption}
% footnotes at bottom
\usepackage[bottom]{footmisc}
% Urls with line break
\usepackage{pdflscape}
% Drawing functions
\usepackage{tikz}
\usepackage{pgfplots}
\pgfplotsset{width=7cm, height=4cm, compat=1.17}

\usepackage{multicol}
\setlength{\columnsep}{1cm}

%%%%%%%%%%%% References/Bibliography %%%%%%%%%%%%
\usepackage{biblatex}
\addbibresource{bibliography.bib}

%%%%%%%%%%%%%%%%%%%%%%%%%%%%%%%%%%%%%%%%%%%%%%%%%
\begin{document}


%%%%%%%%%%%%%%%%%% Cover Page %%%%%%%%%%%%%%%%%%
\title{\vspace{-0.9cm}
       \vspace{1cm}
       \normalsize
       \raggedright\textbf{Title: \hspace{1.5cm} SIO - Report 1} \\ \vspace{0.4cm}
       \raggedright\textbf{Author: \hspace{1.12cm} Eduardo Santos 93107, Pedro Sobral 98491, Pedro Bastos 93150} \\ \vspace{0.4cm}
       \raggedright\textbf{Date: \hspace{1.48cm} 02/03/2023} \\}
\author{}
\date{}

\maketitle
\thispagestyle{fancy}

%%%%%%%%%%%%%%%%%% END Cover Page %%%%%%%%%%%%%%%%%%

\vspace{-1.4cm}

\tableofcontents


\fontsize{10pt}{13pt}
\selectfont
\lsstyle

\titlelabel{\thetitle.\quad}	

%\newpage

\section{Introductory Note}

\section{Application Context of the University of Aveiro}

\section{Structural Features of a CRM}

\section{Specification Booklet to our Use Case}

\section{OSS CRMs on the Market}

\subsection{Odoo}

Odoo CRM is an open source customer relationship management software that is designed to help businesses of all sizes manage their sales, marketing, and customer support. It offers a wide range of features, including lead management, sales pipeline management, marketing automation, and customer service tools.

One of the standout features of Odoo CRM is its modular design. The software is built on an extensible architecture that allows users to mix and match different features and apps to create a customized solution that meets their specific needs. Odoo CRM also integrates with more than 10,000 other apps, making it easy for businesses to connect with other tools in their tech stack.

Odoo CRM includes advanced sales pipeline management tools, such as customizable stages, automated lead assignment, and forecasting capabilities. The software also offers marketing automation features, including email campaigns, lead nurturing workflows, and website visitor tracking.

The software's customer service tools allow businesses to manage customer inquiries and support requests from within the platform. Users can create tickets, track customer interactions, and assign tasks to team members.

Odoo CRM is built on the Python programming language and is available as both a cloud-based and on-premises solution. It also offers mobile apps for iOS and Android devices, allowing users to access the software from anywhere.

Overall, Odoo CRM is a powerful and flexible CRM solution for businesses of all sizes. Its modular design, advanced sales pipeline management tools, and marketing automation features make it a popular choice for businesses looking to improve their customer relationship management processes.

\subsection{OroCRM}

OroCRM is a customer relationship management platform designed for businesses of all sizes, from small startups to large enterprises. It offers a variety of features to help businesses manage their customer interactions, such as contact management, sales pipeline management, marketing automation, and analytics.

One of the standout features of this OSS is its flexibility. It is built on the Symfony2 PHP framework, which is widely used and well-liked among web developers. This means that OroCRM's code is easily accessible to many open-source developers, who can create customizations and extensions to meet individual business needs.

OroCRM's sales pipeline management features are particularly robust. Users can set up specific stages for each sales team, create sub-stages to better organize the sales process, and use a drag-and-drop interface to move opportunities through the pipeline. Lost opportunities are automatically archived, and other opportunities can be manually archived for future reference.

Another useful feature of OroCRM is its marketing automation capabilities. Users can create targeted email campaigns, track website visitor behavior, and manage social media interactions from within the platform.

OroCRM also offers powerful analytics tools, including the ability to track the performance of sales teams and analyze customer behavior patterns. These insights can help businesses make data-driven decisions to improve their customer relationship management processes.

Overall, OroCRM is a flexible, cost-effective solution for businesses seeking to improve their customer relationship management processes. Its open-source nature and accessibility to web developers make it a popular choice for businesses of all sizes.

\subsection{X2CRM}

X2CRM is an open-source Customer Relationship Management (CRM) software that offers a suite of tools to help businesses manage customer interactions, sales, and marketing. It was designed for small to mid-sized businesses looking for an affordable, customizable CRM solution.

The software offers a wide range of features, including contact management, lead generation, email marketing, and sales pipeline management. It also includes advanced reporting and analytics, allowing businesses to track key performance indicators and make data-driven decisions.

One of the unique features of X2CRM is its workflow automation, which allows businesses to automate repetitive tasks and streamline their processes. The software also includes a drag-and-drop interface for building custom workflows and integrations with other applications.

X2CRM is built on the Yii PHP framework, which is known for its speed and scalability. The software is available in both cloud-based and on-premises versions, giving businesses flexibility in how they deploy and manage the software.

Overall, X2CRM is a powerful and customizable CRM solution for businesses looking to improve their customer relationship management processes. Its advanced features, workflow automation, and affordability make it a popular choice for small to mid-sized businesses.

\subsection{Vtiger}

Vtiger CRM is an open-source customer relationship management software that is designed to help businesses of all sizes manage their customer interactions, sales, and marketing. It offers a wide range of features, including contact management, sales pipeline management, marketing automation, and analytics.

One of the standout features of Vtiger CRM is its customization capabilities. The software allows users to create custom modules and fields and includes a drag-and-drop interface for designing custom workflows. This flexibility makes it easy for businesses to tailor the software to their specific needs.

Vtiger CRM also includes robust sales pipeline management tools, such as customizable stages, automated lead assignment, and forecasting capabilities. The software also integrates with popular email and calendar applications, making it easy for sales teams to manage their communications and appointments.

The software's marketing automation features allow users to create targeted email campaigns, track website visitor behavior, and manage social media interactions from within the platform. Vtiger CRM also includes advanced reporting and analytics, allowing businesses to track key performance indicators and make data-driven decisions.

Vtiger CRM is built on the PHP programming language and is available as both a cloud-based and on-premises solution. It also offers mobile apps for iOS and Android devices, allowing users to access the software from anywhere.

Overall, Vtiger CRM is a powerful and customizable CRM solution for businesses of all sizes. Its customization capabilities, sales pipeline management tools, and marketing automation features make it a popular choice for businesses looking to improve their customer relationship management processes.

\subsection{EspoCRM}

\subsection{CiviCRM}

\subsection{Trello}

\subsection{Airtable}

\subsection{Bitrix24}

\subsection{suiteCRM}

\section{CRM Selection Criteria}

\section{Chosen CRM to the University of Aveiro}




\section{Conclusion}


% Add "References" to table of contents
%\addcontentsline{toc}{section}{References}
% No cite makes all references appear, even if there's no citation on the text
\nocite{*}
\printbibliography

\end{document}